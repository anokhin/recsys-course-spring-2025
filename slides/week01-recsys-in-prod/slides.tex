\documentclass[11pt,aspectratio=169,handout]{beamer}

\usetheme{Singapore}
\usecolortheme{orchid}

\usepackage[utf8]{inputenc}
\usepackage[russian]{babel}
\usepackage{amsmath}
\usepackage{amsfonts}
\usepackage{amssymb}
\usepackage{graphicx}
\usepackage{bibentry}
\usepackage{wasysym}
\usepackage[most]{tcolorbox}
\usepackage[normalem]{ulem}

\usepackage{hyperref}

\definecolor{info}{RGB}{62, 180, 137}
\definecolor{warn}{RGB}{128, 0, 0}

\author{Николай Анохин}
\title{Рекомендательные сервисы в продакшене}

\logo{\includegraphics[width=.05\textwidth]{images/ok_logo.png}}

\AtBeginSection[]{
  \begin{frame}
  \vfill
  \centering
  \begin{beamercolorbox}[sep=8pt,center,shadow=true,rounded=true]{title}
    \usebeamerfont{title}\insertsectionhead\par
  \end{beamercolorbox}
  \vfill
  \end{frame}
}

\begin{document}

{
\setbeamertemplate{headline}{}

\begin{frame}
\titlepage
\end{frame}

}

\begin{frame}{}

\begin{center}
\includegraphics[scale=0.35]{images/overall.png}
\end{center}

\vfill

Входной опрос: \url{https://shorturl.at/jvxzY}

\end{frame}

\section{Архитектуры рекомендательных сервисов}

\begin{frame}
\begin{center}
\includegraphics[scale=0.15]{images/mendeley-recs.jpeg}
\end{center}
\end{frame}

\begin{frame}{Обзор типичных компонентов RS / Mendeley (2016) \cite{MNDL}}
% Components
\begin{columns}
\begin{column}{0.6\textwidth}
   \begin{center}
		\includegraphics[scale=0.2]{images/mendeley.jpeg}
   \end{center}
\end{column}
\begin{column}{0.35\textwidth}
    \begin{tcolorbox}[colback=info!5,colframe=info!80,title=]
    \begin{small}
    Машинное обучение -- небольшая часть рекомендательного сервиса. Другие компоненты часто требуют не меньше усилий.
    \end{small}
    \end{tcolorbox}
\end{column}
\end{columns}

\footnote{Ссылки:
\href{https://hadoop.apache.org/}{Hadoop} +
\href{https://spark.apache.org/}{Spark} +
\href{https://hbase.apache.org/}{HBase} +
\href{https://www.elastic.co/what-is/elasticsearch}{Elasticsearch}
}

\end{frame}

%\begin{frame}{Netflix Prize 2007-2009}
%
%\begin{center}
%\includegraphics[scale=0.1]{images/bellkor.jpg}
%\end{center}
%
%\end{frame}

\begin{frame}{Обработка данных под высокой нагрузкой / Netflix (2013) \cite{NFLX}}

\begin{columns}
\begin{column}{0.6\textwidth}
   \begin{center}
		\includegraphics[scale=0.1]{images/netflix.png}
   \end{center}
\end{column}
\begin{column}{0.35\textwidth}
    \begin{small}
    \begin{tcolorbox}[colback=info!5,colframe=info!80,title=]
    Двигаясь от offline к real-time, мы можем быстрее реагировать на изменения контекста. При этом возникают ограничения на сложность алгоритмов.
    \end{tcolorbox}
    \end{small}
\end{column}
\end{columns}

\footnote{Ссылки:
\href{https://cassandra.apache.org/_/index.html}{Cassandra}
}

\end{frame}

\begin{frame}
\begin{center}
\includegraphics[scale=0.25]{images/netflix-6.png}
\end{center}
\end{frame}

\begin{frame}{Рекомендации айтемов из больших каталогов / Youtube (2016) \cite{YTBE}}
% Enormous action space, context
\begin{columns}
\begin{column}{0.6\textwidth}
   \begin{center}
		\includegraphics[scale=0.25]{images/youtube.png}
   \end{center}
\end{column}
\begin{column}{0.35\textwidth}
   \begin{small}
    \begin{tcolorbox}[colback=info!5,colframe=info!80,title=]
    Айтемов так много, что учесть {\bf полный контекст} не может даже Google. Для быстрого отбора кандидатов применяются грубые фильтры.
    \end{tcolorbox}
    \end{small}
\end{column}
\end{columns}

\end{frame}

\begin{frame}{Уровни вычислений и двухэтапный рекомендер \cite{YAN}}

\begin{center}
	\includegraphics[scale=0.3]{images/discovery-system-design.jpg}
\end{center}

\end{frame}

\begin{frame}{Загадка}

Что общего между
\begin{itemize}
\item населением городов
\item количеством друзей у пользователей в социальной сети
\item размерами лесных массивов
\item количеством прослушиваний песен в Spotify
\end{itemize}

\end{frame}

\begin{frame}{Power law}

\[
p(x) = \frac{C} {x^{\alpha}}, \quad x > x_{min}
\]

\begin{center}
\includegraphics[scale=0.25]{images/longtail.png}

Правило 80/20
\end{center}

\end{frame}

\begin{frame}{Холодный старт и длинный хвост / Spotify (2016) \cite{SPTF}}
% Long tail, coldstart
\begin{columns}
\begin{column}{0.6\textwidth}
   \begin{center}
		\includegraphics[scale=0.31]{images/spotify.jpeg}
   \end{center}
\end{column}
\begin{column}{0.35\textwidth}
\begin{small}
    \begin{tcolorbox}[colback=info!5,colframe=info!80,title=]
    Холодные айтемы и пользователи будут всегда. Использование контента - один из вариантов решения проблемы
    \end{tcolorbox}
\end{small}
\end{column}
\end{columns}

\footnote{Ссылки: \href{https://kafka.apache.org/}{Кafka}}

\end{frame}

\begin{frame}{Несоответствие таргетов моделей и business-value / TikTok (2020) \cite{TIK}}
% Intangible objectives, content, exploration/authors!
\begin{columns}
\begin{column}{0.6\textwidth}
   \begin{center}
		\includegraphics[scale=0.24]{images/tiktok.png}
   \end{center}
\end{column}
\begin{column}{0.35\textwidth}
    \begin{small}
    \begin{tcolorbox}[colback=info!5,colframe=info!80,title=]

    Потребности людей нельзя упаковать в удобную метрику. Кроме машинного обучения  в рекомендательных сервисах приходится исполтзовать пре- и пост-процессинг, чтобы гарантировать business-value.

    \end{tcolorbox}
    \end{small}
\end{column}
\end{columns}

\end{frame}

\begin{frame}{Дата-платформы}

\begin{columns}
\begin{column}{0.6\textwidth}
   \begin{center}
		\includegraphics[scale=0.25]{images/ok-feed.png}
	\end{center}
\end{column}
\begin{column}{0.35\textwidth}
    \begin{small}
    \begin{tcolorbox}[colback=info!5,colframe=info!80,title=]
    Данные -- основной ресурс RS, поэтому важно иметь надежное и понятное логирование. МЛ -- двигатель, но модели сложно менеджить и анализировать. Поэтому {\bf дата платформы} -- это база.
    \end{tcolorbox}
    \end{small}
\end{column}
\end{columns}

\footnote{Ссылки: \href{https://mlflow.org/}{mlflow} + \href{https://airflow.apache.org/}{Airflow}}

\end{frame}

\begin{frame}{Как в действительности выглядит архитектура RS}

\begin{center}
\includegraphics[scale=0.15]{images/bosch.jpeg}
\end{center}

\end{frame}

\begin{frame}{Какие сложности учитывает архитектура RS}

\begin{itemize}
\item Высокая нагрузка рекомендательных сервисов
\item Большие каталоги айтемов
\item Холодный старт пользователей и айтемов
\item Несоответествие business-value и метрик оптимизации
\end{itemize}

\end{frame}

\begin{frame}{Какие технические средства могут понадобиться}

\begin{itemize}
\item Отказоустойчивые продакшен-сервисы (HBase, Cassandra, Elasticsearch)
\item Передача данных (kafka)
\item Хранение данных (Hadoop HDFS)
\item Batch обработка данных (Spark)
\item Потоковая обработка данных (Kafka, Spark Streaming)
\item MLOps тулы (Airflow, MLFlow)
\end{itemize}

\end{frame}

\section{Метрики и эксперименты}

\begin{frame}{Netflix 2010-2021 \cite{NETFLIXAB}}

\begin{columns}
\begin{column}{0.45\textwidth}
   \begin{center}
		\includegraphics[scale=0.15]{images/netflix-2010.png}
   \end{center}
\end{column}
\begin{column}{0.45\textwidth}
   \begin{center}
		\includegraphics[scale=0.12]{images/netflix-2021.png}
   \end{center}
\end{column}
\end{columns}

\end{frame}

\begin{frame}{}

Хотим принимать решения на основе данных $\rightarrow$ 

\qquad Начинаем собирать метрики $\rightarrow$ 

\qquad \qquad Разрабатываем инструменты для принятия решений

\begin{center}
\includegraphics[scale=0.18]{images/uber.png}
\end{center}

\footnote{Ссылки: ELK = Elasticsearch + \href{https://www.elastic.co/kibana/}{Kibana}}

\end{frame}

\begin{frame}{Наивный подход к измерению эффекта}
\begin{center}
\includegraphics[scale=0.3]{images/noexp.png}
\end{center}
\end{frame}

\begin{frame}{}

\begin{tcolorbox}[colback=info!5,colframe=info!80,title=Задача]
Какой {\bf причинно-следственный эффект} на распределение целевой метрики окажет выбранное воздействие $T$?
\end{tcolorbox}

\vfill

\begin{tcolorbox}[colback=warn!5,colframe=warn!80,title=Фундаментальная Проблема Causal Inference]
Для конкретного пользователя невозможно вычислить causal effect напрямую, потому что нельзя пронаблюдать значение целевой переменной при более чем одном значении $T$\footnote{Без дополнительных предположений эту проблему не решить \cite{GELMAN}}
\end{tcolorbox}

\end{frame}

\begin{frame}{Randomized Controlled Experiment}

\begin{tcolorbox}[colback=info!5,colframe=info!80,title=Схема эксперимента]
Все доступные пользователи независимо друг от друга случайным образом распределяются в control либо treatment с одинаковой вероятностью
\end{tcolorbox}

\begin{tcolorbox}[colback=gray!5,colframe=gray!80,title=Предположение 1: ]
Можно оценить значение некоторой характеристики для всей популяции, имея выборку из этой популяции.
\end{tcolorbox}

\begin{tcolorbox}[colback=gray!5,colframe=gray!80,title=Предположение 2: Stable Unit Treatment Value Assumption]
Эффект для каждого пользователя зависят только от свойств этого пользователя, но не свойств и исходов других пользователей.
\end{tcolorbox}

\end{frame}

\begin{frame}{Фреймворк Potential Outcomes}

Воздействие на $i$ пользователя:
\[
T_i = \begin{cases}
0, \quad \text{если показываем сontrol} \\
1, \quad \text{если показываем treatment}
\end{cases}
\]

Соответствующие потенциальные исходы:
\[
y_i^0 \text{ и } y_i^1
\]

Требуется оценить:
\begin{tcolorbox}[colback=info!5,colframe=info!80,title=Average Treatment Effect,center,width=6cm,center title]
\[
ATE = E \left[ y_i^1 - y_i^0 \right]
\]
\end{tcolorbox}

\end{frame}

\begin{frame}{Оцениваем ATE в RCE}

\[
ATE = E[y_i^1 - y_i^0] = E[y_i^1] - E[y_i^0] \sim \text{avg}_{i \in T}(y_i^1) - \text{avg}_{i \in C}(y_i^0) = \bar y_1 - \bar y_0
\]

\begin{itemize}
\item нужно оценить две характеристики -- $E[y_i^0]$ и $E[y_i^1]$, поэтому используем выборки $C$ и $T$
\item проще всего сделать оценку, если выборка несмещенная
\item чем больше данных, тем точнее оценка
\end{itemize}

\end{frame}

\begin{frame}{}

\begin{center}
\includegraphics[scale=0.3]{images/yesexp.png}
\end{center}

\end{frame}

\begin{frame}{Доверительный интервал на ATE}

Доверительный интервал $(L, U)$ с уровнем доверия $\alpha$:
\[
P(L < \theta < U) = 1 - \alpha
\]

Формула Уэлча:
\[
\bar y_1 - \bar y_0 \pm t_{\alpha/2,r} \sqrt{\frac{s_1^2}{n_1} + \frac{s_0^2}{n_0}}, \quad
r = \frac{ \left( \frac{s_1^2}{n_1} + \frac{s_0^2}{n_0} \right)^2 }{ \frac{s_1^4}{n_1^2 (n_1 - 1)} + \frac{s_0^4}{n_0^2 (n_0 - 1)} }
\]
Где:
\begin{itemize}
\item $n_1$ и $n_0$ -- количество пользователей в treatment и control
\item $s_1^2$ и $s_0^2$ -- оценки дисперсии метрики в treatment и control
\item $t_{\alpha/2,r}$ -- табличное значение для $r$ степеней свободы
\end{itemize}

\end{frame}

\begin{frame}{На практике}

\begin{itemize}

\item Перед запуском
\begin{itemize}
\item Выбираем ключевую метрику, несколько сопутствующих метрик и контролируем, что не ``уронили'' важные
\item Выбираем длительность эксперимента, оценивая мощность теста :D
\end{itemize} 

\item При анализе
\begin{itemize}
\item Метрики распределены по-разному: нужно подбирать подходящие тесты
\item Используются методы снижения дисперсии оценок (cuped, diff-in-diff)
\end{itemize}

\end{itemize}

\vfill

\begin{tcolorbox}[colback=info!5,colframe=info!80]
Если вы попали в компанию, в которой есть культура принятия решений на основе данных -- сохраняйте ее всеми силами. Если нет -- пропагандируйте.
\end{tcolorbox}

\end{frame}

\begin{frame}{Сложности RCE в индустриальных рекомендерах \cite{FACEBOOK}}

\begin{itemize}[<+->]
\item Как выбрать Самую Главную Метрику (OEC)? 
\item Как оценить долгосрочный эффект?
\item Разный эффект на разных сегментах пользователей
\item Отсутствие культуры экспериментирования
\item Масштабирование платформы для экспериментов
\item Сетевые эффекты
\item Наложение эффектов от экспериментов
\end{itemize}

\end{frame}

\section{Итоги}

\begin{frame}

\begin{tcolorbox}[colback=info!5,colframe=info!80]
В основе рекомендательных сервисов лежит машинное обучение. При проектировании  нужно учитывать множество дополнительных факторов, например требования к скорости обработки данных, эффект длинного хвоста и возможность холодного старта.
\end{tcolorbox}
\vfill
\begin{tcolorbox}[colback=info!5,colframe=info!80]
A/B эксперимент -- надежный способ оценки эффекта от изменений в сервисе.
\end{tcolorbox}

\end{frame}

\begin{frame}{Подпишис}

\begin{columns}
\begin{column}{0.45\textwidth}
   \begin{center}
		\includegraphics[scale=0.35]{images/natasha.png}
   \end{center}
\end{column}
\begin{column}{0.45\textwidth}
   \begin{center}
		\url{https://t.me/mlvok}
		
		\includegraphics[scale=0.5]{images/tgqr.png}
   \end{center}
\end{column}
\end{columns}

\end{frame}

\begin{frame}
\begin{center}
\includegraphics[scale=0.4]{images/thanks.png}
\end{center}

\end{frame}

\begin{frame}[allowframebreaks]{Литература}
\bibliographystyle{amsalpha}
\bibliography{../references.bib}
\end{frame}

\end{document}
